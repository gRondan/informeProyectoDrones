\chapter{Introducción}

El proyecto propone el estudio de distintos algoritmos colaborativos de inteligencia computacional basados en el paradigma de computación basada en agentes para determinar los patrones de exploración sobre un conjunto de vehículos (flota) capaces de operar de forma autónoma. La exploración es realizada sobre un escenario conocido a priori, teniendo los vehículos información sobre las áreas notables del mismo como son la presencia de puntos de interés y zonas inaccesibles para transitar. A su vez también se propone como objetivo simultáneo a la exploración la vigilancia de ciertos puntos del territorio previamente marcadas para dicho fin, llamados “Puntos de Interés” (POI por sus siglas en inglés), los cuáles pueden tener distinto tipo de prioridad de vigilancia asignados al iniciar la misión. Cabe destacar que la colaboración de los vehículos es llevado a cabo directamente entre los vehículos involucrados en la flota sin comunicarse con una estación base. Los vehículos utilizados en la investigación son vehículos aéreos no tripulados (UAV por sus siglas en inglés) conocidos comúnmente como drones. Al utilizar drones se tiene una limitante determinante para la exploración que es el tiempo de duración de una carga de batería. Al ser un trabajo colaborativo otra de las limitantes existentes es la calidad de la comunicación entre los drones, la cual puede verse afectada por diferentes aspectos como la distancia entre los mismos, condiciones atmosféricas, interferencias, etc. El objetivo de la exploración a realizar es el de cubrir la máxima proporción de territorio, realizarlo de forma tal de optimizar el tiempo de la misión debido principalmente a la limitante del tiempo de duración de la batería de los drones y con la capacidad de que los drones puedan continuar realizando la misión aunque hayan perdido la conectividad con el resto de la flota.
La motivación de este proyecto viene dada por el auge de la utilización de drones, principalmente los conocidos como vehículos aéreos no tripulados en diversos tipos de tareas y entornos. Dentro de las principales aplicaciones para el uso de drones no tripulados se encuentran las desarrolladas para tareas agrícolas [1], tanto en fertilización de campos como en mediciones, análisis del estado del suelo, de los cultivos, etc. También se encuentran aplicaciones muy controvertidas pero que cada vez son más utilizadas e importantes que son las aplicaciones militares [2]. Los drones no tripulados en aplicaciones militares son utilizados desde misiones de reconocimiento de zonas peligrosas, vigilancia [3] y apoyo en tareas de rescate  [4] hasta misiones de ataque de objetivos específicos (contando con drones que poseen armamento). 
Si bien existen distintos trabajos y estudios que refieren a la coordinación de drones para la realización de tareas, la mayoría de las aplicaciones que son llevadas a la práctica constan de drones pilotados de forma remota [4], donde es necesario contar con un piloto profesional certificado para conducir los vuelos, por lo que aún existe un amplio margen para estudiar soluciones a problemas donde los drones operen de forma autónoma y que puedan ser aplicables en la realidad. Para alcanzar los objetivos planteados en el estudio se analizan y comparan los comportamientos de los distintos algoritmos colaborativos mediante la ejecución de pruebas en simuladores. También se implementan y ejecutan los programas con los algoritmos desarrollados directamente en la flota de drones a modo de poder observar su comportamiento en la realidad.
En este documento se presentan los antecedentes encontrados y estudiados de forma previa a la realización del proyecto. A continuación se detallan las características de los drones utilizados, tanto a nivel de hardware como de software. Luego se describe la parte central del informe. En la misma se detallan las características de todas las herramientas utilizadas para el desarrollo del proyecto y se describe la máquina de estados implementada para la solución al problema. En la siguiente sección se describe la metodología para la experimentación, de describen los distintos casos de pruebas planteados y se analizan los resultados obtenidos. Finalmente el informe finaliza con el planteo de las conclusiones obtenidas en base a los resultados de la experimentación, incluyendo posibilidades de trabajo a futuro teniendo como punto de partida la investigación realizada.
