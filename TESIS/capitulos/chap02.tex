\chapter{Revisión de antecedentes}

Existen diversos antecedentes de investigaciones para la coordinación de un equipo de UAVs con el fin de completar tareas, motivados principalmente por fines militares.  En general estos antecedentes centran la formulación del problema en la monitorización de uno o varios objetivos determinados a priori como es el caso del siguiente estudio:

Frank Mufalli, Rajan Batta y Rakesh Nagi (2012)  [5]realizaron una investigación titulada “Simultaneous Sensor Selection and Routing of Unmanned Aerial Vehicles for Complex Mission Plans” que busca resolver el problema de optimizar tareas de investigación de instalaciones fijas, donde se conocen de antemano sus coordenadas, alternando entre los diferentes tipos de sensores con los que se equipan a las unidades UAV con las que se dispone para cumplir la misión, dado que cada instalación a investigar produce mejores o peores beneficios en función de que sensor lleva equipado el drone que la investigue.
El artículo utiliza como base para su desarrollo el TOP (team orienteering problem), el cual a su vez es una generalización de OP (orienteering  problem), que busca solucionar el problema de un vehículo que partiendo de un origen debe llegar a un destino previamente establecido.

Si bien los drones con los que se cuenta para el proyecto son similares, los mecanismos de selección para determinar cuál dron de la flota debe encargarse de un punto de interés en base a cuál obtiene el mayor beneficio puede adaptarse al problema planteado ya que los puntos de interés tendrán distintas prioridades lo que determina en cierto punto un beneficio mayor el de atender un punto con mayor prioridad.
Otro aspecto fundamental del proyecto es el tratamiento de las limitantes propias de los drones como lo son la duración de la batería y las comunicaciones entre los mismos. En base a esto encontramos los siguientes antecedentes de estudios: 

Cesare (2015)  [6] en su trabajo “Multi-UAV Exploration with Limited Communication and Battery” propone un algoritmo para la coordinación de un equipo de UAVs con el fin de explorar un territorio previamente desconocido, teniendo en cuenta las limitaciones que surgen por el tiempo de vuelo limitado de los drones por su batería y de los problemas que pueden surgir por problemas en la red que se utiliza para la comunicación del equipo. El mismo se basa en la premisa de que es aceptable que al finalizar la exploración no es necesario que todos los drones retornen al punto de partida.
El algoritmo definido por el autor se basa en una máquina de estados compuesta de cinco estados, siendo los mismos denominados explore, meet, sacrifice, relay y go home. Explore es el estado inicial de todos los UAV que forman parte del equipo y es aquí donde se define la estrategia de exploración del dron basándose la misma en definir fronteras entre el espacio explorado y el espacio desconocido. Después de un tiempo constante previamente definido se cambia al estado meet, donde el dron busca comunicarse con otro miembro del equipo para transmitir su información y a su vez actualizar la información interna que posee sobre las posiciones del resto de los drones; a su vez al comunicarse con otro miembro se negocia con el mismo quien pasa al estado relay y quien al estado sacrifice. Si se determina que el dron será sacrificado pasa al estado de exploración hasta que se calcule que le queda solo el tiempo suficiente de vuelo como para volver a encontrarse con el dron que se determinó que estará en estado relay, es entonces cuando cambia su estado a sacrifice y busca volver a encontrarse con el dron de estado relay para transmitirle la información que recolecto. En cambio el dron que fue seleccionado para después pasar a estado relay continúa su viaje volviendo al estado exploración, y cuando determina que le queda tiempo de vuelo suficiente solo para volver a la base aterriza para esperar al dron de estado sacrifice y finalmente retornar a la base cambiando al estado go home.

Este artículo provee un interesante enfoque sobre una arquitectura distribuida para la exploración de un territorio que en muchos puntos es similar al problema que se busca resolver en este proyecto, teniendo como principal diferencia la característica que se permite que no todos los miembros del equipo de UAVs retornen a la base.

Esten Ingar Grøtli y Tor Arne Johansen [7] en su trabajo “Path planning for UAVs under communication constraints using SPLAT! and MILP” desarrollaron un algoritmo basado en MILP para resolver el problema del offline path planning para múltiples UAVs. MILP es un algoritmo que optimiza una serie de variables basado en un sistema de ecuaciones y restricciones. En este paper el objetivo de los autores era desplegar los drones en una zona conocida para que sirvan de relés de comunicación entre estaciones aisladas. La parte central del algoritmo desarrollado se basa en resolver un MILP. Tomando en cuenta sus objetivos los autores definieron una función objetivo que determina rutas para los drones optimizando la conectividad entre los UAVs de la flota y minimizando el consumo de combustible. En resumen, los cuatro componentes principales de la función objetivo describen: el consumo de combustible para mantener la velocidad crucero, el consumo de combustible para acelerar y frenar, el consumo de combustible para cambiar de altitud y el nivel de conectividad entre los UAV de la flota. A su vez, las restricciones definidas en el MILP se usan para evitar colisiones, salirse de las zonas de vuelo autorizadas, entre otros controles. Este algoritmo de optimización es ejecutado en sucesivas iteraciones, en cada una alimentándose con datos de SPLAT!, un servicio para comunicaciones por radio que permite estimar el nivel de conectividad entre dos puntos en el espacio, lo cual permitió generar rutas progresivamente mejores con cada iteración. El proceso es repetido hasta que las rutas calculadas por el algoritmo cumplan con los requisitos de la misión que tengan que cumplir los drones, los cuales tienen que ser definidos de alguna forma matemática.
Los autores probaron su algoritmo con simulaciones y mostraron que produce rutas efectivas que se ajusta a una gran variedad de situaciones y obstáculos.

Si bien este artículo no está tan enfocado en la exploración sino en distribuir efectivamente los UAVs de una flota en un terreno, el enfoque de usar el MILP resulta muy interesante ya que puede ser adaptado a una multitud de situaciones incluyendo la de optimizar la exploración.

Ke Shang et al. (2014) [8] en su investigación titulada “A GA-ACO Hybrid Algorithm for the Multi-UAV Mission Planning Problem” desarrollaron un algoritmo híbrido que sirve para solucionar el problema de misiones de vigilancia con flotas de UAV, donde el problema se caracteriza dentro de uno de optimización combinacional. El algoritmo busca la planificación de rutas para los UAVs de modo que poder obtener el máximo beneficio en la vigilancia satisfaciendo la limitante de energía de los UAV. Se le considera algoritmo híbrido ya que utiliza la combinación de dos tipos diferentes de algoritmos: un algoritmo genético (GA, algoritmo evolutivo conocido) y un algoritmo de optimización de una colonia de hormigas (ACO). El problema está compuesto por objetivos de vigilancia los cuales deben ser visitados por la flota de UAV, donde cada uno de los objetivos provee de un beneficio de vigilancia determinado según su importancia. Las dos métricas utilizadas para medir la eficiencia y eficacia del algoritmo son la suma total de beneficios de vigilancia obtenidos y el tiempo de la misión. El problema planteado es modelado por los autores mediante un grafo conectado donde los nodos son los objetivos a visitar (0=nodo inicial, n=nodo final), cada arista entre nodos representa un camino entre ellos y donde cada objetivo tiene asociado un beneficio de vigilancia particular. El objetivo del algoritmo es encontrar m caminos empezando y terminando desde los nodos inicial y final visitando los objetivos de modo de maximizar el beneficio de vigilancia. Cada nodo se puede visitar como máximo una sola vez. La restricción aplicada es un tiempo máximo de duración de la misión que depende de la energía restante de los UAV de la flota. 
En cuanto a la especificación del algoritmo híbrido implementado por los autores, se distinguen responsabilidades distintas para cada algoritmo de forma individual: mientras que el algoritmo genético es el encargado de intentar mantener y evolucionar una población (en este caso de nodos objetivos a visitar) hacia una mejor solución, el ACO iterativamente permite que las hormigas (en este caso los elementos de la flota de UAV) dejen un rastro con feromonas para encontrar mejores soluciones. El algoritmo propuesto incorpora ambas habilidades de mejoras de los dos algoritmos anteriores donde la innovación está dada por la suplantación de malos individuos de la población por nuevos generados por el ACO la cual mejora la calidad de la solución a medida que se incrementan las iteraciones. Los resultados obtenidos mediante simulación muestran que el algoritmo planteado ofrece beneficios similares o mejores que los máximos ofrecidos por los restantes algoritmos comparados mientras que en cuanto al tiempo ofrece un tiempo promedio aceptable para su aplicabilidad.

Con el estudio anterior se pueden obtener ciertos aspectos definidos los cuales pueden utilizarse en el proyecto como por ejemplo la forma de caracterizar la eficiencia y eficacia del territorio a recorrer más allá del cubrimiento. Un aspecto diferencial entre ambos estudios es que el realizado por Ke Shang et al. (2014) no considera el problema de la limitante en la conexión entre los drones.

En relación al problema específico del manejo de la comunicación entre los drones de la flota resaltamos los siguientes estudios:

Schleich et al. (2013) [9] en su trabajo “UAV Fleet Area Coverage with Network Connectivity Constraint” desarrollaron un modelo de cubrimiento de áreas para una flota de drones no pilotados con la restricción de mantener la mayor conectividad posible entre los mismos y con una estación base. El modelo introducido se basa en la selección de las rutas a tomar por los drones mediante el comportamiento por feromonas. Los autores introducen cuatro métricas para la evaluación de su modelo y comparan los resultados obtenidos contra un modelo de selección aleatoria de las rutas. El trabajo resalta la capacidad que tiene una flota de drones en misiones de reconocimiento, exploración y vigilancia ya que ofrece teóricamente la posibilidad de duración ilimitada de la misma ya que un subconjunto de la flota puede estar recargando sus energía mientras el resto continúa con la misión. El escenario propuesto implica el mantenimiento de una red ad-hoc entre los drones de la flota así como también la obtención de medidas de forma simultánea del área a explorar en al misión. Las cuatro métricas introducidas por los autores con las que se comparan los distintos modelos son: velocidad de cubrimiento del área, exhaustividad del cubrimiento del área, confiabilidad del cubrimiento y el mantenimiento de la conectividad. Para obtener valores para las métricas se divide el área a explorar en pequeños rectángulos de largo y ancho determinado de modo que cada vez que un dron pasa por esa área se manda un aviso a los drones conectados y a la base. El modelo de cubrimiento de área desarrollado cuenta de tres etapas secuenciales: selección del dron vecino al cual se intenta estar conectado, cálculo de la ruta alternativa viable para dónde debe dirigirse y la aplicación del comportamiento basado en feromonas para seleccionar la mejor ruta basado en los cálculos anteriores. Los resultados obtenidos por los autores en cuanto a las métricas medidas fueron buenos para el modelo presentado cuando la flota de UAV´s tenía un número generalmente mayor a 20 nodos. En cuanto a conectividad entre la flota y porcentaje de nodos conectados a la base obtuvo mejores resultados que los otros modelos, manteniéndose relativamente similar en cuanto al resto de las métricas.
Si bien el estudio ataca una de las limitantes principales del proyecto como lo es la conectividad entre los drones, en este caso se evalúa la calidad de la conexión en relación a cuántos drones están conectados a una estación base, característica que no se tiene en la investigación actual. En el caso de las métricas planteadas por los autores para evaluar la calidad de la misión, las mismas son aplicables al proyecto actual adaptando la medida de conectividad al contexto planteado.

I. Bekmezci et al. (2013) [10] , en su investigación “Flying Ad-Hoc Networks (FANETs): A survey, Ad Hoc Network.”, presentan en su artículo una nueva denominación para la familia de redes ad hoc encargada de la comunicación en sistemas multi UAV el cual denominan “Flying ad hoc Network (FANET)”. También presentan una definición para este tipo de familia, una comparación con familias existentes como lo son las MANETs (Mobile ad hoc Network) y VANETs (Vehicle ad hoc Networks), desafíos en el diseño de las FANET, escenarios donde aplique el uso de FANET, distintos protocolos utilizados en ellas, investigaciones abiertas, etc, mediante la recopilación de distintos artículos de la literatura. A través de esta recopilación de la literatura existente sobre las FANET, los autores presentan estudios que focalizan en distintas áreas de las FANET como las características de diseño de las mismas, consideraciones al diseñarlas y distintos tipos de protocolos existentes en las FANET. Finalmente se mencionan estudios donde se generen resultados a través de simulaciones de las distintas áreas mencionadas.
Inicialmente se presentan los problemas que surgen al utilizar un pequeño grupo de UAVs para una misión particular. Entre los más importantes está el problema del diseño la infraestructura necesaria para la comunicación entre los distintos UAVs. Los autores mencionan dos enfoques utilizados en la literatura: comunicación a través de una infraestructura entre los UAV con una estación base y otro enfoque de comunicación entre los distintos UAV directamente mediante una red ad hoc generada entre ellos. Las características principales de los elementos de una FANET según la literatura presentada son las siguientes: gran movilidad de los nodos y gran distancia entre ellos, topología de la red muy cambiante y nodos que deben soportar tanto una comunicación peer-to-peer como también recolectar información del entorno mediante sensores. En el artículo se mencionan distintos escenarios en donde se puede aplicar el concepto de una FANET. Entre ellos se destacan escenarios donde se desee extender la escalabilidad de una misión con multi UAVs en los que generalmente no existe o no se puede comunicar con una estación base (debido a las grandes distancias entre los nodos y la base por ejemplo), escenarios donde se necesite una comunicación multi UAV confiable, escenarios de enjambres de UAVs (para evitar colisiones entre ellos por ejemplo), etc.
Dentro de las características en el diseño de una FANET, se mencionan los siguientes puntos: nodos con gran movilidad y con gran velocidad lo que lleva a que las FANET tengan que recalcular continuamente las rutas de vuelo de los UAV, baja densidad de los nodos con mucha distancia entre ellos, cambios rápidos y continuos en la topología de la red (debido a la movilidad y/o fallas en los nodos), poder computacional en general limitado por el tamaño de los UAV, localización en general a través de GPS en cada UAV, etc.   
Luego los autores hacen referencia a estudios donde se focalizan las consideraciones a tener en cuenta para el diseño de una FANET, de las que se destacan: adaptabilidad a entornos diversos y cambiantes, escalabilidad para poder aplicar los algoritmos con cualquier cantidad de UAVs, latencia mínima en el envío de paquetes entre los UAV, limitantes impuestas por el hardware de las UAV y un ancho de banda necesario para el envío de la información obtenida por los UAV.
Como último ítem importante los autores destacan los distintos protocolos de comunicación en una FANET recopilados en la literatura. Se observa que se dividen de acuerdo a la capa de red donde aplican los protocolos. Encontramos en la capa física dos tipos de modelos: el modelo de propagación de ondas de radio y el modelo de estructura de antena. Tenemos en la capa MAC principalmente el protocolo IEEE 802.11 como el más utilizado. En la capa de red se observa de los estudios que se utilizan generalmente protocolos de ruteo provenientes de las MANET. Sin embargo una estrategia de ruteo basada solamente en información de la locación de los nodos puede satisfacer las necesidades de una FANET. En la capa de transporte los estudios mencionan que la responsabilidad más grande de los protocolos es la de la confiabilidad en el transporte de los paquetes, control de flujo y control de congestión de los mismos.
Los estudios mencionados en la sección de simulaciones presentan como las distintas implementaciones de soluciones para resolver los problemas típicos de las FANET llevan a resultados aceptables, dejando también muchos campos de investigación abiertos.

El artículo otorga una visión global del problema de diseño de una red ad hoc entre UAVs, el estado actual de investigación en el área y permite relevar algunos estudios realizados en áreas específicas del problema planteado, permitiendo un mayor entendimiento de a que se enfrenta cuando se quiere diseñar una red ad hoc entre UAV. Se considera este artículo de gran valor por la gran recopilación de la literatura disponible.
Como último antecedente que trata sobre el problema de la comunicación entre drones dejamos planteados los principales conceptos desarrollados en el capítulo 66 del libro “Handbook of Unmanned Aerial Vehicles” escrito por los autores Andrew Kopeikin, Sameera S. Ponda y Jonathan P. How [11] .


El capítulo busca analizar las tecnologías existentes sobre control de comunicación por red en equipos de UAVs. Dado que los UAVs son móviles, las comunicaciones en estas redes son casi exclusivamente inalámbricas y en general buscan transmitir la información hacia una central donde la información se procesa. También se busca en general que los diferentes UAVs puedan comunicarse entre ellos de forma que puedan actuar en conjunto. La calidad de una red se puede evaluar en función a su topología, es decir su configuración de enlaces entre nodos y la calidad de estos. Existen varios mecanismos de control que permiten mejorar la calidad de una red de UAVs, la mayoría se traducen básicamente a implementar una o varias de estas acciones:
Posicionar los UAVs de forma que se reduzca la distancia entre ellos
Aumentar la potencia de los transmisores, o eligiendo ubicaciones especialmente favorables para la transmisión
Eligiendo activamente las conexiones habilitadas en una red y determinando caminos óptimos dentro de la red
Aumentando la cantidad de UAVs en la red

Los principales desafíos que una red inalámbrica tiene que superar son la distancia entre los nodos y la interferencia provocada ya sea por obstáculos físicos o por el ruido de otras señales. Los UAVs aéreos en general funcionan en cielo abierto y eso les permite evitar la mayoría de estos problemas, aunque no siempre. Hay situaciones donde los UAVs deben volar en zonas urbanas o a baja altitud y eso acarrea problemas. De esta forma el objetivo que se le quiere dar a la red y el tipo de vehículo que se va usar en ella es fundamental para decidir su diseño. La potencia de los transmisores del UAV puede verse limitada por el consumo de energía que representan, por el espacio que ocupan, o porque otros instrumentos del UAV pueden ser de mayor prioridad (incluyendo la CPU).
La arquitectura de la red depende en gran parte del tipo de información que se quiere enviar a través de ella. Una red de UAVs debe tener esto en cuenta ya que puede usarse para transmitir una gran variedad de datos, aunque básicamente se los puede subdividir en dos clases:
Mensajes de control y comandos. Estos pueden dar órdenes a los UAVs, transmitir telemetría básica sobre el estado del UAV (como su posición, cantidad de batería, etc). Estos mensajes requieren poco ancho de banda pero no son resistentes a fallos y pueden requerir un bajo delay.
Información captada por los sensores del UAV. Estos pueden ser imágenes, video, audio, datos complejos, etc. Requieren un mayor ancho de banda y tienen los mismos requerimientos que si fueran a ser transmitidos por Internet.

La topología de una red puede definirse fácilmente como un grafo G=(V,E) pero también como una matriz de adyacencia A tal que aij = 1 si hay una conexión entre los nodos i y j, y 0 en otro caso. Una de las representaciones más comunes es la de la matriz Laplaciana: 
Donde A es la matriz de adyacencia y D es la matriz diagonal tal que:
Los valores propios de esta matriz cumplen con la propiedad:
Y en particular  es conocido como la conectividad algebraica del grafo y determina la velocidad de convergencia en la mayoría de los algoritmos de toma de decisiones por consenso, por lo que muchos algoritmos intentan maximizar este valor. De todas formas hay que indicar que orientar el algoritmo de consenso únicamente a maximizar  no resulta una buena idea ya que se corre el riesgo de que los UAVs se centren demasiado en mejorar la conectividad entre ellos y no en cumplir con su verdadero objetivo.
