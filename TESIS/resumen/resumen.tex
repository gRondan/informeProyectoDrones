\begin{abstract}

El presente trabajo constituye el estudio de algoritmos colaborativos de exploración y misiones de vigilancia sobre puntos predefinidos mediante una flota de drones capaces de operar de forma autónoma sobre escenarios con diversas características. La implementación de la solución al problema se basa en el paradigma de la computación basada en agentes. La misma está constituida por una máquina de estados encargada de resolver las distintas etapas por las que atraviesan los drones cuando ejecutan la misión, teniendo como objetivos el máximo cubrimiento de la zona a explorar en un tiempo óptimo de acuerdo a la duración de la batería de los drones, manteniendo vigilancia sobre puntos de interés previamente definidos e intentando mantener comunicación constante entre los drones de la flota. Una de las principales características de la solución implementada es que la misma es ejecutada por cada dron de forma autónoma, sin ninguna estación base que esté calculando rutas a seguir ni ningún otro tipo de decisión. El lenguaje de programación utilizado fue Python, las comunicaciones se manejan a través de conexiones TCP por medio de una red WiFi. Se utilizaron drones Parrot Bebop 2 para la investigación. Si bien la solución tiene en cuenta cuestiones específicas de las características de este modelo de dron, la misma es adaptable a cualquier otro.
Con el objetivo de validar el algoritmo implementado para la solución del problema se realizaron pruebas sobre distintos escenarios de exploración de áreas abiertas mediante simulaciones computacionales utilizando el simulador Sphinx. Se realizó la ejecución del algoritmo para diversas instancias de cada escenario con una flota de dos drones. También se realiza la ejecución de distintos tipos de algoritmos de modo de poder comparar los resultados experimentales. Como complemento se realiza la instalación y ejecución del algoritmo directamente en cada uno de los drones de modo de comprobar su correcto comportamiento en la realidad.
Los resultados obtenidos en la experimentación 
FALTA RESULTADOS GENERALES DE LOS RESULTADOS
La investigación concluye 
FALTA CONCLUSIONES GENERALES 
El trabajo realizado contribuye a presentar un desarrollo aplicable en la práctica que le permite a una flota de drones poder ejecutar tareas de exploración y vigilancia de puntos predefinidos de forma colaborativa y autónoma, maximizando el cubrimiento del territorio así como también optimizando la vigilancia sobre los puntos de interés elegidos.


\end{abstract}

