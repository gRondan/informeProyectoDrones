\chapter{Conclusiones}

Dadas las limitantes encontradas en el Parrot Bebop 2 se destaca el hecho de haber sido posible ejecutar código de forma nativa por medio del lenguaje de scripting Python, generando una versión portable del mismo que puede ser trasladada a demás drones y dispositivos que cuenten con la arquitectura ARMv7 en su CPU. Esto no solo expande las posibilidades del trabajo realizado por el equipo, sino también de la comunidad de desarrolladores que a partir de este sistema pueden expandir las aplicaciones implementadas que se ejecutan directamente en los drones.
A su vez se encontró la dificultad de no poder implementar la conectividad de los drones por medio de una red ad-hoc que era uno de los objetivos propuestos al iniciar el trabajo. El obstáculo que impidió cumplir esta meta fue la limitación que otorga el dron a modificar su configuración por defecto. A pesar de esto se consiguió desarrollar una alternativa por medio de establecer un canal de comunicación entre los drones a través de una red wifi que se considera que se adapta al objetivo inicialmente planteado, destacando la facilidad de uso para el usuario final dado que es suficiente con presionar tres veces el botón de apagado del dron para modificar el modo de la interfaz wifi y conectarse a una red configurada.
La facilidad de instalación de las soluciones provistas tanto para ejecutar código en el dron como para establecer la conexión a la red wifi genera que el proceso de implantación del sistema en un dron que se encuentre en estado de fábrica sea extremadamente sencillo, dado que el mismo consiste en la copia de archivos al dron y la ejecución de un script. Esto puede resultar extremadamente útil si se desea utilizar una flota cuantiosa de drones para realizar la exploración de un territorio.
Sin embargo el principal problema encontrado fue el de conseguir determinar la ubicación del dron durante la ejecución del sistema. La alternativa empleada durante la investigación fue la de utilizar un posicionamiento relativo del dron en función de los movimientos realizados, lo cual no consigue los resultados de precisión esperados. Se analizaron otras posibilidades pero ninguna llegó a mejorar los resultados obtenidos con la solución seleccionada y por este motivo fueron descartados. Se concluye que el mayor problema reside en la falta de sensores equipados en el Parrot Bebop 2 que permitan realizar mediciones de distancia con respecto a objetos. La mejor alternativa en términos de precisión alcanzados, es la que incluye el procesamiento de las imágenes recabadas por la cámara del dron pero esto implica un costo extra en términos de recursos de hardware del dron que no se ha demostrado que sea capaz de soportar. Sí existen en cambio soluciones que realizan este procesamiento de forma remota haciendo uso de una base central que procese las imágenes con mayor capacidad de cómputo que los drones.
FALTA CONCLUIR EVALUACIÓN EXPERIMENTAL

